Quando se conduz uma revisão de literatura sem o pré-estabelecimento de um protocolo de revisão há um direcionamento por interesses pessoais, o que leva a resultados pouco confiáveis. Neste contexto, pesquisadores vem utilizando uma técnica denominada a Revisão Sistemática (RS) para auxiliar o pesquisador a conduzir um revisão bibliográfica de forma totalmente sistemática com o intuito de evitar que trabalhos importantes fiquem fora de suas pesquisas. Uma RS é caracterizada por ser um meio de avaliar e interpretar todas as pesquisas disponíveis, referentes a um questão de pesquisa, tema, área ou fenômeno de interesse. A RS tem como objetivo apresentar uma avaliação justa de um tema de pesquisa, utilizando uma metodologia confiável, rigorosa e auditável~\cite{kit04}.

De acordo com~\citet{kit04} a RS implica na forma mais adequada para se identificar, avaliar e interpretar toda pesquisa importante para um tema em particular. Resume-se que uma RS configura um alicerce para novas atividades de pesquisa acerca de determinado tema. Dessa forma, realizou-se uma RS com o intuito de aferir o estado da arte dos assuntos desse trabalho de doutorado.

Neste cenário, o objetivo da RS é efetuar um levamento bibliográfico para caracterizar quais técnicas de mineração de interesses tranversáis têm sido utilizadas durante a fase de modernização de sistemas legados. Para atingir este objetivo, foi realizado uma RS. Os resultados, bem como o protocolo de pesquisa elaborado, renderam uma publicação no evento ACM SAC 2013, o qual tem qualis A1. Esse artigo será apresentado pelo bolsista em Março de 2013 em Coimbra, Portugal. Maiores detalhes sobre esse artigo pode ser obtido no Apêndice B deste relatório. O mesmo apêndice contem uma cópia da obra publicada.