Durante o período ao qual se refere esse relatório, o bolsista realizou a elaboração do texto para o exame de qualificação exigido pelo Programa de Pós-Gradução em Ciências de Computação e Matemática Computacional. A realização da Revisão Sistemática, a escolha das disciplinas oferecidas pelo Programa, publicação e participações em eventos científicos foram fundamentais para o bom aproveitamento deste atividade.

O texto da qualificação possui um total de 4 Capítulos. O primeiro capítulo da qualificação contêm as seguintes seções: (\textit{i}) \textbf{Contextualização}, que apresenta um resumo do trabalho a ser desenvolvido - (\textit{ii}) \textbf{Motivação}, onde são descritas as motivações para a realização do trabalho a ser desenvolvido - (\textit{iii}) \textbf{Objetivos}, a qual descreve os principais objetivos do projeto de doutorado - e, (\textit{iv}) \textbf{Organização}, que descreve a organização do texto da qualificação. 

O segundo capítulo consiste das seguintes seções: (\textit{i}) \textbf{Considerações Iniciais}, que descreve o objetivo do capitulo - (\textit{ii}) \textbf{Modernização Dirigida à Arquitetura}, onde são descritos os principais conceitos sobre essa abordagem - (\textit{iii}) \textbf{Reengenharia de Software}, a qual descreve todos os conceitos sobre reengenharia de software, bem como os conceitos de refatoração - (\textit{iv}) \textbf{Desenvolvimento Dirigido a Modelos}, descreve os conceitos relacionados a tal abordagem - (\textit{v}) \textbf{Considerações Finais}, o qual conclui e descreve o conteúdo que foi apresentado nesse capítulo. 

O terceiro capítulo descreve e apresenta vários trabalhos relacionados com o projeto de pesquisa. E o quarto capítulo contêm as seguintes seções: (\textit{i}) \textbf{Considerações Iniciais}, que descreve o objetivo do capitulo - (\textit{ii}) \textbf{Proposta de Projeto}, onde descreve os objetivos do trabalho a ser desenvolvido pelo bolsista - (\textit{iii}) \textbf{Desafios de Pesquisa com Relação ao Projeto}, apresenta os principais desafios que o bolsita possivelmente irá enfrentar durante o desenvolvimento dos objetivos do trabalho - (\textit{iv}) \textbf{Metodologia}, descreve os passos que o bolsista irá realizar para realizar os objetivos do trabalho - (\textit{v}) \textbf{Avaliação}, apresenta como o bolsista irá avaliar o trabalho realizado - (\textit{vi}) \textbf{Resultados Esperados}, descreve os principais resultados esperados ao término do trabalho - (\textit{vii}) \textbf{Colaborações}, descreve as principais colaborações que o bolsista tem com outros pesquisadores - (\textit{viii}) \textbf{Considerações Finais}, conclui e descreve o conteúdo que foi apresentado nesse capítulo.

É importante destacar que a apresentação da qualificação foi realizada no dia 12 de Dezembro de 2012. Tal apresentação resultou na aprovação do projeto proposto.