A seguir são descritas as atividades que o bolsista pretende conduzir durante o próximo período de vigência da bolsa.

\begin{enumerate}
	\item \textbf{Implementação da abordagem de modernização:} no próximo período de vigência do bolsista, o mesmo irá estudar qual a melhor forma para implementar a abordagem de modernização de sistemas legados. A princípio a implementação constituirá de um \textit{plug-in} do Eclipse\footnote{www.eclipse.org}. Nesse \textit{plug-in} o engenheiro de software entrará com um sistema legado que almeja-se ser modernizado. Assim, a abordagem implementada irá modernizar tal sistema utilizando técnicas de MDD, ADM e KDM (ver Seções~\ref{sub:estudo_sobre_reengenharia_de_software}, \ref{sub:estudo_sobre_desenvolvimento_dirigido_a_modelos} and \ref{sub:estudo_sobre_moderniza_o_dirigida_arquitetura}).

	\item \textbf{Redação de artigos:} pretende-se continuar produzindo artigos, com o intuito de documentar o progresso do projeto em questão. O próximo passo consiste em publicar um experimento avaliando a implementação que será realizada. Uma possível abordagem para condução do experimento seria constratar a abordagem implementada pelo bolsista com uma outra abordagem já disponível na literatura.

	\item \textbf{Avaliação:} após a implementação da abordagem proposta pelo bolsista, pretende-se avaliar o ambiente de modernização resultante. Pretende-se realizar dois tipos de experimentos. O primeiro consiste em realizar um conjunto de estudos de casos para investigar a viabilidade da abordagem proposta, bem como avaliar o uso das funcionalidades do apoio computacional para fornecer suporte à modernização de sistemas legados. O segundo experimento consistem em realizar avaliações controladas utilizando a metodologia experimental~\cite{Wohlin}, a fim de avaliar o impacto da abordagem proposta, bem como do apoio computacional relacionado a eficiência e impacto das equipes e também a qualidade em termos de modularidade, reuso e manutenibilidade dos sistema resultantes durante a atividade de modernização. 
\end{enumerate}