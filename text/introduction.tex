 Este relatório tem por objetivo apresentar as atividades realizadas pelo bolsista Rafael Serapilha Durelli durante o período de Junho/2012 a Março/2013,
 referente à bolsa de doutorado concebida pela Fundação de Amparo à Pesquisa do Estado de São Paulo (FAPESP) sob o Processo Número 2012/05168-4. 
 
 É importante salientar que o trabalho em questão tem sido desenvolvido no Departamento de Ciências da Computação e Estatística do Instituto de Ciência 
 Matemáticas e de Computação (ICMC) da Universidade de São Paulo (campos São Carlos/SP).  
 Este trabalho se insere no contexto do grupo de pesquisas em Engenheira de Software, sob a orientação do Prof. Dr. Márcio Eduardo Delamaro.  
 Além disso, este trabalho esta sendo executado em colaboração com o grupo de engenharia de software da Universidade Federal de São Carlos (UFSCAR)\footnote{http://dc.ufscar.br}.  
 Mais especificamente em colaboração com o Prof. Dr. Valter Vieira de Camargo\footnote{http://buscatextual.cnpq.br/buscatextual/visualizacv.do?id=S819089},
 o qual tem grande experiência na área de engenharia de software com ênfase no desenvolvimento de frameworks no contexto da programação orientada a aspectos e reuso de software. 

Por fim, ressalta-se que o bolsista criou um vínculo científico com o \textit{Institut National de Recherche en Informatique et en Automatique} (INRIA),  
onde irá realizar um ano de doutorado sanduíche sobre orientação do  
 Prof. Dr. Nicolas Anquetil\footnote{http://rmod.lille.inria.fr/web/pier/team/Nicolas-Anquetil} o qual tem grande experiência na área de manutenção e reengenharia de software.