

A proposta do Desenvolvimento Dirigido a Modelos (\textit{Model-Driven Development - MDD}) é reduzir a distância semântica entre o problema do domínio e a solução/implementação. Assim, o engenheiro de software não precisa interagir inteiramente com o código-fonte, podendo-se concentrar em modelos que possuem maiores níveis de abstração. Um mecanismo é responsável por gerar automaticamente o código-fonte por meio dos modelos. No MDD, modelos não apenas guiam as tarefas de desenvolvimento e manutenção, mas são partes integrante do software assim como o código-fonte, servindo como entrada para ferramentas de geração de código reduzindo os esforços dos desenvolvedores~\cite{Bittar, Kleppe:2003}


No desenvolvimento tradicional, ou seja, sem seguir o MDD, artefatos de alto nível (ex., modelos, diagramas) usualmente são produzidos antes da codificação e costumam ser úteis apenas nas etapas iniciais do ciclo de desenvolvimento. Conforme o desenvolvimento evolui, mudanças são aplicadas somente no código-fonte e não nos modelos/diagramas. Assim, tais artefatos acabam se tornando incoerentes, ou seja, não refletem o que o código-fonte apresenta, o que faz com que o tempo e os esforços gastos na construção desses artefatos não sejam diretamente aproveitados na produção do software~\cite{Bittar}. Contrapartida, o MDD tem o foco nos modelos e busca simplificar o processo de desenvolvimento de software. Modelos são mais intuitivos para representação do conhecimento e menos dependentes do código-fonte, de forma que podem ser reutilizados facilmente em diferentes projetos. Diferentemente, o código-fonte possui uma linguagem que é densa e codificada, tornando-se difícil identificar, extrair e reutilizar o conhecimento apenas pela leitura do mesmo~\citep{Kleppe:2003}.



% subsection desenvolvimento_dirigido_a_modelos (end)