Nesta seção são apresentadas as publicações, participações em eventos científicos e apresentações de trabalho realizadas pelo bolsista durante o primeiro período de vigência da bolsa de doutorado. 

Por enquanto, o resultado mais relevente do projeto é proveniente de uma revisão sistemática conduzida pelo bolsista com o intuito de identificar e obter conhecimento da área de pesquisa. Assim, um artigo sumarizando as observações dessa revisão sistemática foi submetido e aceito para o evento ACM SAC 2013, o qual possui Qualis A1. Tal evento será realizado em Coimbra, Portugal e será apresentado pelo bolsista em Março/2013. O artigo é denominado \textit{A Systematic Review on Mining Techniques for Crosscutting Concerns} e encontra-se em anexo a este relatório. A seguir são descritos outras publicações que foram conduzidas como atividades complementares e não estão diretamente relacionados ao projeto de pesquisa do bolsista. Vale resssaltar que tais artigos foram desenvolvidos em parceria com outros membros do grupo de pesquisa. Tais artigos foram aceitos e apresentados por um dos autores. Além disso é importante salientar que não foi usado nenhum recurso disponível em Reserva Técnica para tais artigos, tais como: inscrição, viagem ou hospedagem.



\begin{itemize}
	\item \textbf{Trabalhos completos publicados em anais de congressos}
	\begin{itemize}
	 	\item DURELLI, R. S. ; DURELLI, V. H. S. . A Systematic Mapping Study on Formal Methods Applied to Crosscutting Concerns Mining. In: IX Experimental Software Engineering Latin American Workshop (ESELAW), 2012, Buenos Aires. IX Experimental Software Engineering Latin American Workshop (ESELAW), 2012~\footnote{\label{note1}Vale ressaltar que esse artigo é uma atividade complementar e não relacionada ao projeto de pesquisa. Esse artigo foi desenvolvido em parceria com outros membros do grupo de pesquisa. Tal artigo foi aceito e apresentado por um dos autores. Além disso é importante salientar que não foi usado nenhum recurso disponível em Reserva Técnica para inscrição, viagem ou hospedagem do bolsista.}.
	 	\item DURELLI, R. S. ; DURELLI, V. H. S. . F2MoC: A Preliminary Product Line DSL for Mobile Robots. In: Simpósio Brasileiro de Sistemas de Informação (SBSI), 2012, São Paulo. Simpósio Brasileiro de Sistemas de Informação (SBSI), 2012~\footref{note1}.
	 	\item Gottardi ; DURELLI, R. S. ; PASTOR, O. L. ; CAMARGO, V. V. . Model-Based Reuse for Crosscutting Frameworks: Assessing Reuse and Maintainability Effort. In: Simpósio Brasileiro de Engenharia de Software, 2012, Natal. Simpósio Brasileiro de Engenharia de Software, 2012~\footref{note1}.
	 	\item DURELLI, R. S. ; Gottardi ; CAMARGO, V. V. . CrossFIRE: An Infrastructure for Storing Crosscutting Framework Families and Supporting their Model-Based Reuse. In: XXVI Simpósio Brasileiro de Engenharia de Software - XXVI Sessão de Ferramenta, 2012, Natal. Simpósio Brasileiro de Engenharia de Software, 2012. v. 6. p. 1-6~\footnote{\label{note2}Vale ressaltar que esse artigo é uma atividade complementar e não relacionada ao projeto de pesquisa. Esse artigo foi desenvolvido em parceria com outros membros do grupo de pesquisa. Tal artigo foi aceito e apresentado por um dos autores}.
	 	\item DURELLI, R. S. ; SANTIBANEZ, D. S. M. ; ANQUETIL, N. ; DELAMARO, M. E. ; CAMARGO, V. V. . A Systematic Review on Mining Techniques for Crosscutting Concerns (to appear). In: ACM SAC 2013, 2012, Coimbra. ACM SAC Software Engineering (SE) Track, 2013. v. 28th~\footnote{Foi usado recurso disponível em Reserva Técnica para pagar a inscrição desse evento. Outras despesas como passagem áerea e diários foram solicitadas e adquiridas pela Programa de Apoio à Pós-Graduação (PROAP).}.
	 \end{itemize}

	 \item \textbf{Resumos expandidos publicados em anais de congressos}

	 \begin{itemize}
	 	\item DURELLI, R. S. ; SANTIBANEZ, D. S. M. ; DELAMARO, M. E. ; CAMARGO, V. V. . A Systematic Review on Mining Techniques for Crosscutting Concerns. In: 6th, Latin American Workshop on Aspect-Oriented Software Development Advanced Modularization Techniques, 2012, Natal. Latin American Workshop on Aspect-Oriented Software Development Advanced Modularization Techniques, 2012
	 \end{itemize}

	 \item \textbf{Apresentação de Trabalho}

	 \begin{itemize}
	 	\item DURELLI, R. S. ; Gottardi ; CAMARGO, V. V. . CrossFIRE: An Infrastructure for Storing Crosscutting Framework Families and Supporting their Model-Based Reuse. Trabalho apresentado no XXIII Simpósio Brasileiro de Engenharia de Software - XVIII Sessão de Ferramentas, 2012, Natal - RN - Brasil (Apresentação de Trabalho/Simpósio).
	 \end{itemize}

\end{itemize}