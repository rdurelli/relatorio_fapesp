%% LyX 2.0.1 created this file.  For more info, see http://www.lyx.org/.
%% Do not edit unless you really know what you are doing.
\documentclass[english]{article}
\usepackage[T1]{fontenc}
\usepackage[utf8]{luainputenc}
\usepackage{array}

\makeatletter

%%%%%%%%%%%%%%%%%%%%%%%%%%%%%% LyX specific LaTeX commands.
%% Because html converters don't know tabularnewline
\providecommand{\tabularnewline}{\\}

\makeatother

\usepackage{babel}
\begin{document}
\begin{tabular}{|>{\centering}p{3cm}|>{\raggedright}p{12cm}|}
\hline 
Vantagem & Caracterização\tabularnewline
\hline 
\hline 
Produtividade & \begin{itemize}
\item Automatização da geração de código a partir de modelos por meio de
uso de ferramentas de transformação, incentivando os desenvolvedores
a retornarem às etapas iniciais de requisitos e análise; 
\item Tarefas repetitivas de codificação são implementadas nas transformações,
poupando tempo e esforço que podem ser despendidos em outras tarefas.\end{itemize}
\tabularnewline
\hline 
Portabilidade & \begin{itemize}
\item A partir de um mesmo modelo pode-se gerar código para diferentes plataformas.\end{itemize}
\tabularnewline
\hline 
Interoperabilidade & \begin{itemize}
\item Cada parte do modelo pode ser transformado em código para uma plataforma
diferente, o que resulta em um software que pode ser executado em
um ambiente heterogêneo, mas que mantém sua funcionalidade global.\end{itemize}
\tabularnewline
\hline 
Manutenção e documentação & \begin{itemize}
\item Alterações são realizadas diretamento nos modelos, mantendo-os consistentes
como o código-fonte, o qual é gerado automaticamente a partir de transformações
aplicadas nesses modelos;
\item A documentação permanece atualizada, o que facilita as tarefas de
manutenção.\end{itemize}
\tabularnewline
\hline 
Comunicação & \begin{itemize}
\item Uma vez que os modelos são mais abstratos que o código-fonte, o que
não exige conhecimento técnico associado à plataforma de implementação
para sua compreensão, os especialistas do domínio podem utilizar diretamente
os modelos para identificar mais facilmente as questões associadas
ao negócio;
\item Os especialistas de tecnologia da informação podem identificar os
elementos técnicas usando os mesmos modelos.\end{itemize}
\tabularnewline
\hline 
Reuse & \begin{itemize}
\item O reuso é feito em nível de modelos ao invés de um nível de código-fonte;
\item O código-fonte pode ser automaticamente regenerado para novos contextos
por meio de ferramentas de transformações apropriadas.\end{itemize}
\tabularnewline
\hline 
Verificação e Otimizações & \begin{itemize}
\item Os modelos facilitam a análise por verificadores semânticos, conforme
a sintaxe de seu metamodelo, e a execução de otimizações automáticas;
\item Minimização da ocorrência de erros semânticos, o que fornece implementações
mais eficientes.\end{itemize}
\tabularnewline
\hline 
Correção & \begin{itemize}
\item Ferramentas de transformação evitam a introdução de erros acidentais,
tais como erros de digitação e de sintaxe;
\item Erros conceituais podem ser identificados em um nível mais alto de
abstração.\end{itemize}
\tabularnewline
\hline 
\end{tabular}
\end{document}
